\documentclass{article}
\usepackage{graphicx}
\usepackage[french]{babel}
\title{Rapport de projet r\'eseau}
\author{DELAR Emmanoe, RAKOTOARIJAONA Camille}
\begin{document}
\maketitle
\begin{figure}[!b]
		\centering
		\includegraphics[height=4cm]{logo.jpg}
	\end{figure}
\newpage
\tableofcontents

\newpage
\section{Introduction}
Lors de ce projet, r\'ealis\'e en bin\^omes, nous avions pour objectif de d\'evelopper un jeu de bataille navale en r\'eseau. Pour cela, nous avons utilis\'e le langage de programmation objet Python.
On est parti du code source fourni par nos enseignants. Ce code nous permettait de jouer contre la machine uniquement. D\`es lors, nous avons d\^u l'am\'eliorer afin de pouvoir jouer \`a 1 contre 1 sur le r\'eseau.

\subsection{Pr\'esentation du jeu}
	 La bataille navale est un jeu de soci\'et\'e dans lequel deux joueurs doivent placer des « navires » sur une grille tenue secrète et tenter de « toucher » les navires adverses. Le gagnant est celui qui parvient \`a torpiller compl\`etement les navires de l'adversaire avant que tous les siens ne le soient.


\section{Fonctionnement du programme}
	 \subsection{Jouer contre la machine}
	 Apr\`es avoir lanc\'e le serveur, lorsqu'on lance le programme avec le num\'ero d\textquoteright h\^ote de celui-ci en argument ( \guillemotleft main ::\guillemotright \ ), si on choisi de ne pas jouer en r\'eseau alors le jeu commence directement. Quand le jeu commence, on choisi la colonne et la ligne \`a viser. Quand c'est au tour de l'ordinateur, une fonction fait de m\^eme en choisissant des coordonn\'ees al\'eatoirement et joue son coup. Et ainsi de suite jusqu'\`a la fin de la partie. 
	 \newline On tient \`a pr\'eciser que l'address non sp\'ecifi\'ee 0:0:0:0:0:0:0:0 ou :: signifie que le port respectif n'est pas en mode \guillemotleft \'ecoute \guillemotright \ a une adresse pr\'ecise mais toute. 

	 \subsection{Jouer en r\'eseau local}
	 Si on choisi de jouer en r\'eseau, comme demand\'e dans le sujet, il faut d’un c\^ot\'e, d\'emarrer le serveur en lan\c cant le programme \guillemotleft main.py\guillemotright \ sans arguments, et du c\^ot\'e client, il faut lancer le programme avec l’adresse ip du serveur en argument.
	 \newline Lorsque ceux-ci sont lanc\'es, le client aura la possibilit\'e de choisir d\textquoteright abord si il veut se connecter au serveur, ou si il veut jouer seul face au robot comme d\'ecrit au dessus. Si il d\'ecide de jouer en r\'eseau, le programme attendra qu\textquoteright une autre personne se connecte au serveur pour qu'ils puissent jouer ensemble.
	 \newline Ensuite, chacun jouera \`a tour de r\^ole jusqu\textquoteright a ce que la partie soit termin\'ee, lorsque la partie est termin\'ee chaque joueur verra le score, et le serveur se d\'econnectera et le jeu de chaque joueur aussi. 
	 \newline Il est \`a noter que les joueurs peuvent donc jouer tout seul peu importe le moment sans affecter le serveur lorsqu\textquoteright ils jouent en local.

\section{Structure}
	Contrairement \`a la partie contre l'ordinateur, en mode r\'eseau, les coordonn\'ees choisies par le joueur client doivent \^etre envoy\'es au serveur pour qu\textquoteright il puisse les retransmettre \`a l'adversaire. Nous avons alors cr\'e\'e un protocole de communication serveur/client TCP sur le port 7777. Ce port nous permettra de recevoir ou d'\'emettre des informations.

	\subsection{Serveur}
	La programmation du serveur \`a \'et\'e un peu plus complexe car il a pour t\^ache de faire le lien entre les joueurs. 
	\newline Au lancement du programme, nous nous retrouvons dans la fonction principale \guillemotleft main \guillemotright\ en attente de demande de connexion, tout cela se fait par la m\'ethode
	\guillemotleft.select\guillemotright \ vu en cours. 
	Lorsque 2 joueurs se connectent, le serveur leur envoie \`a l\textquoteright aide de la m\'ethode \guillemotleft socket.send \guillemotright , leur num\'ero, 0 ou 1 selon leur ordre de connexion. Ce num\'ero d\'efini celui d'entre eux qui jouera le premier.
	\newline La deuxième \'etape, nous a demand\'e beaucoup plus de r\'eflexion car nous avons choisi d'envoyer le jeu cr\'e\'e par le serveur, aux clients connect\'es.
	\newline En effet, cr\'eer un jeu al\'eatoire depuis le serveur puis l\textquoteright envoyer aux clients nous offre beaucoup de possibililit\'ees. Par exemple impl\'ementer l\textquoteright extension qui nous permet d'avoir des spectateurs devient plus simple car nous pouvons donc leur envoyer ce m\^eme jeu.
	\newline Pour parvenir \`a cela, nous avons d\^u  modifier la fonction \guillemotleft randomConfiguration\guillemotright \ qui g\'en\'ere al\'eatoirement une table de bateaux.
	\newline Dans cette fonction, tout se passe normalement, puis \`a chaque fois que l\textquoteright on g\'en\'ere les coordonn\'ees et la position d\textquoteright un navire, on stocke ces informations dans un tableau. Une fois la g\'en\'eration termin\'ee, on envoie le contenu du tableau, bit \`a bit (en utilisant la m\'ethode \guillemotleft client.send \guillemotright \ ), aux joueurs/spectateur(s) connect\'es avec une boucle qui parcourt le tableau.
	\newline Cette partie nous a pos\'ee plusieurs probl\`emes, par exemple, au niveau de la taille des bateaux, les valeurs envoy\'es devaient \^etre de type \guillemotleft bytes\guillemotright\ . La taille de ces variables pouvait varier, de 1 \`a 2 bytes car les valeurs g\'en\'er\'ees par la fonction  \guillemotleft randomConfiguration\guillemotright\ allaient de  1 \`a 10 et 10 est repr\'esent\'e sur 2 bytes. On a donc d\'ecid\'e de d\'ecaler de 1 chaque valeur pour avoir un interaval de 0 \`a 9 et ainsi avoir une taille au maximum de 1 byte pour chaque valeurs envoy\'ees. La valeur re\c cue puis convertie en entier de l\textquoteright autre c\^ot\'e par le(s) joueurs/spectateur(s) serait incr\'ement\'ee de un. 
	\newline Un autre des probl\`emes complexe que nous avons fix\'e \`a l'aide du d\'ebogage, provenait du fait que la fonction \guillemotleft isValidConfiguration\guillemotright\ retournait de base plusieurs valeurs aléatoirement sans avoir au pr\'ealable v\'erifi\'e si c'\'etait une configuration valide pour notre instance de jeu. Nous avons alors compris qu\textquoteright il fallait modifier cette fonction afin qu'elle retourne les configurations valide et envoie le tout aux clients connect\'es.
	\newline Comme la fonction de base, la nouvelle fonction retourne un navire. On r\'ep\`ete alors l'op\'eration deux fois pour avoir notre table de jeu. Ensuite, c\textquoteright est le d\'ebut de la partie, la boucle qui fait jouer le serveur est la m\^eme que celle du robot et des joueurs. Cependant, celle-ci se place d’abord dans la partie du premier joueur et attendra qu\textquoteright il envoie son coup. Il verra le coup s\textquoteright afficher sur sa table de jeu et l\textquoteright enverra \`a son adversaire, tout cela avec la m\'ethode \guillemotleft recv\guillemotright\ et \guillemotleft.send\guillemotright\ . Le tour passera ensuite au deuxi\`eme joueur puis la boucle fera de m\^eme et ainsi de suite jusqu’à ce que le jeu soit fini. Une fois le jeu termin\'e, le programme sort de la boucle et lib\`ere les sockets puis ferme le serveur.

	\subsection{Connexion client}
	Pour que le client se connecte au serveur, il nous faut le nom de l'h\^ote et le num\'ero de port.
	\newline Ensuite, on va cr\'eer une socket (client) de type ipv6 qui va nous permettre d'ouvrir une connexion avec une machine locale et d'\'echanger des informations (\`a l'aide de la m\'ethode  \guillemotleft .bind\guillemotright\ ).
	\newline Pour r\'ecuperer les premieres informations, notamment le num\'ero du joueur et les navires, on intercepte les premiers bytes envoy\'es par la socket serveur (\`a l'aide de la m\'ethode \guillemotleft client.recv\guillemotright\ ).
	\newline Avec ces informations, le client peut afficher la partie g\'en\'er\'ee par le serveur gr\^ace \`a la fonction Game.
	\newline Une fois la partie cr\'ee puis lanc\'ee, si c'est au joueur actuel de jouer, il entre les coordonn\'ees de la grille adverse \`a viser. Ensuite, on va envoyer ces coordonn\'ees (x,y) sous forme de paquets de bytes au serveur. Le serveur va retransmettre ces paquets de byte \`a l'autre joueur (\`a l'aide de la m\'ethode \guillemotleft client.send\guillemotright\ ).
	\newline Par contre si c'est au tour de l'autre joueur, la socket client se met en mode r\'eception et attend les coordonn\'ees choisies par l'adversaire (\`a l'aide de la m\'ethode \guillemotleft client.recv\guillemotright\ ). Ces donn\'ees sont \'egalement envoy\'ees par le serveur. 
	\newline Les donn\'ees re\c cu sont converties en coordonn\'ee de type entier puis passer en arguments de la fonction \guillemotleft addShot \guillemotright\ . Cette fonction tente de viser la flotte ennemie \`a ces coordonn\'ees. Si le tire est juste, l'adversaire voit la case correspondante barr\'ee, sinon la case est entour\'ee pour repr\'esenter un \'echec.    
	\newline Ce protocole est r\'ep\'et\'e jusqu'\`a la fin de la partie.

	\subsection{Protocole de communication}
	Une fois la connexion \'etablie, la communication serveur/client est tr\`es simple.
	Chaque information n\'ecessaire \`a l'initialisation d'une partie est envoy\'e par le serveur bit \`a bit et ainsi r\'ecup\'er\'e du c\^ot\'e client.
	Il faut noter la n\'ecessit\'e \`a convertir les bytes recus en entier (int).
\newpage
\section{Extensions}
	\subsection{Joueur vs robot}
	Pour impl\'ementer l\textquoteright extension \guillemotleft jouer vs robot \guillemotright \ , il nous a juste fallu mettre une condition et si elle est respect\'ee, on lance le programme initiale qui simule d\'eja une partie joueur/robot. 
	
	\subsection{Observateurs}
	L'extension \guillemotleft Observateurs \guillemotright \ nous a demand\'ee une fonction en plus. Cette fonction d\'efinie l'observateur comme un joueur en plus et initialise sa nouvelle table de jeu. Ce n'est qu'au moment d'afficher la table de jeu que \c ca change. Notre fonction \guillemotleft observator \guillemotright va demander au serveur d'afficher deux tables \`a la fois \guillemotleft displayGame(game,0) \guillemotright et \guillemotleft displayGame(game,1) \guillemotright respectivement la table du joeur 0 puis celle du joueur 1 . \`A la fin, cette fonction teste \'egalement si le jeu est termin\'e et lequel des joueurs l'a remport\'e puis l'affiche. Et ainsi, nous avons un spectateur qui voit les coups jou\'es par les deux joueurs.


\section{Conclusion}
Sur une p\'eriode d'un semestre, les diff\'erentes notions sur l'organisation g\'en\'erale des protocoles r\'eseaux vues en cours nous ont permis de r\'ealiser ce projet. \`A l'aide de la biblioth\`eque standard \guillemotleft socket \guillemotright \ du langage de programmation objet Python. Cette interface de programmation bas-niveau permet aux applications de communiquer en passant directement par la couche transport du mod\`ele OSI. Nous avons alors cr\'e\'e un serveur de jeu bataille navale en r\'eseau.
\newline Tout au long du projet, nous sommes tomber sur de nombreux probl\`emes par rapport au mode de communication de ce syst\`eme. Par exemple pour que deux sockets communiquent, il faut \'echanger des paquets de bytes et non pas des entiers. Nous avons aussi vu que ...
\newline En prenant du recul, nous pouvons dire que le projet nous a apport\'e un bagage suppl\'ementaire. Nous avons consolid\'e nos connaissances g\'en\'erales sur la programmation r\'eseau bas niveau, la couche transport du systeme OSI et avons aussi appris \`a coder \`a un niveau sup\'erieur.

\end{document}